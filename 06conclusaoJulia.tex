% Prof. Dr. Ausberto S. Castro Vera
% UENF - CCT - LCMAT - Curso de Ci\^{e}ncia da Computa\c{c}\~{a}o
% Campos, RJ,  2021
% Disciplina: Paradigmas de Linguagens de Programa\c{c}\~{a}o
%



\chapter{Conclus\~{o}es}
%Neste trabalho apresentamos a linguagem Julia de forma resumida, 
%?pensando em um público familiarizado com os conceitos básicos de programação. Caso seja seu primeiro contato com a programação, recomendamos:

Nesse trabalho abordamos a linguagem Julia, sua história, principais aplicações, sintaxe e ecosistema de uso, entretanto devido as limitações de tempo foi necessário uma abordagem breve, e voltada a pessoas que já tenham certa familiaridade com programação, entretato podemos recomendar para os iniciantes o livro tal, no qual abordará pontos que não foram considerados aqui.

Aos interessados em aprender a linguagem recomendamos o livro Think Julia, para se aprofundar na linguagem o artigo de seus criadores é um excelente ponto de partida. 

Assim temos na linguagem Julia um lingugaem interessante, com ótimas propostas, limitações bem definidas e conhecidas, que pde acrescemtar miotp 

Também recomendamos os conteúdos digitais, indexados informalmente no youtube, que são as conferências da linguagem e as iniciativas de demonstração, podemos destacar o curso do Grant Sanderson, os workshops para data Sciecne, o canal talented beguiners .

Também a plataforma exercism para prática e real aprendizado, que se dá na prática.




Os problemas enfrentados neste trabalho ...
O trabalho que foi desenvolvido em forma resumida ...
Aspectos n\~{a}o considerados que poderiam ser estudados ou \'{u}teis para ...



   \begin{figure}[H]
    \begin{center}
        \caption{Aplica\c{c}\~{a}o da Linguagem R} \label{ling2}
        \includegraphics[width=12cm]{R02.png} \\
        {\tiny \sf Fonte: O autor }
    \end{center}
   \end{figure} 