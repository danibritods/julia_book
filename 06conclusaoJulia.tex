% Prof. Dr. Ausberto S. Castro Vera
% UENF - CCT - LCMAT - Curso de Ci\^{e}ncia da Computa\c{c}\~{a}o
% Campos, RJ,  2021
% Disciplina: Paradigmas de Linguagens de Programa\c{c}\~{a}o
%



\chapter{Considerações Finais}
%Neste trabalho apresentamos a linguagem Julia de forma resumida, 
%?pensando em um público familiarizado com os conceitos básicos de programação. Caso seja seu primeiro contato com a programação, recomendamos:

Nesse trabalho nós apresentamos uma breve introdução à linguagem Julia. Abordamos um pouco de sua história, suas principais aplicações, sua sintaxe e estrutura. Também apresentamos aplicações para exemplificar diferentes aspectos da linguagem e finalmente demonstramos as principais ferramentas utilizadas em seu ecossistema. 

É importante ressaltar que apesar de suas notáveis qualidades, Julia, como qualquer outra linguagem, também tem pontos negativos importantes de se considerar. Jakob Nybo Nissen em seu artigo de 2021: "What's bad about Julia?"\cite{Nissen2021} mapeia as principais limitações da linguagem, dentre as quais podemos citar as duas que são relevantes em nosso escopo.

A primeira limitação da linguagem é a latência da primeira execução.Isto é, assim que executamos qualquer código Julia pela primeira vez precisamos esperar de segundos a minutos para que seja compilado. Apenas as execuções subsequentes têm a velocidade que lhe é característica. Tal fato virtualmente inviabiliza o uso de Julia em pequenos scripts Unix por exemplo, bem como em aplicações que a responsividade seja sempre chave.

Ademais, a segunda limitação está no fato de aplicações escritas em Julia, mesmo as mais simples, consumirem no mínimo 150MB memória RAM, o que restringe ainda mais as possíveis aplicações para mobile, sistemas embarcados, processos daemon dentre outros.  

Finalmente, devido a proposta desse trabalho nos atemos aos aspectos básicos, e apenas mencionamos por vezes o mecanismo da linguagem e construções mais complexas. 
Dentre os aspectos não considerados temos principalmente questões da engenharia de software em Julia. Como por exemplo os testes de software, os padrões de design da linguagem, medição de performance, o uso de macros. 

Aos interessados em aprender a linguagem recomendamos o livro Think Julia \cite{Lauwens2019}, o excelente canal do Youtube \href{https://www.youtube.com/c/juliafortalentedamateurs}{"Julia for talented Amateurs"} e principalmente o ótimo site e comunidade de aprendizado \href{https://exercism.org/tracks/julia}{Exercism.com} que trás trilhas de aprendizagem para as mais diversas linguagens (inclusive Julia) por meio de exercícios divertidos, com dicas e correções da comunidade. Além de um crescente ecossistema de conteúdo como o curso "Computational Thinking" do MIT (18.S191 MIT Fall 2020) \footnote{https://computationalthinking.mit.edu/Spring21/}\footnote{https://github.com/mitmath/18S191} disponibilizado no \href{https://www.youtube.com/playlist?list=PLP8iPy9hna6Q2Kr16aWPOKE0dz9OnsnIJ}{Youtube}.

Já para se aprofundar na linguagem, o melhor ponto de partido é o seu artigo oficial, onde seus criadores detalham seu design e mecanismo. \cite{Bezanson2017}

Assim temos em Julia um linguagem com ótimas propostas, limitações bem definidas, e uma comunidade ativamente trabalhando para expandi-la e aprimorá-la. O futuro dirá até onde sua ambição é capaz de chegar. 
