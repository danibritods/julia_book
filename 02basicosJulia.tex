% Prof. Dr. Ausberto S. Castro Vera
% UENF - CCT - LCMAT - Curso de Ci\^{e}ncia da Computa\c{c}\~{a}o
% Campos, RJ,  2021
% Disciplina: Paradigmas de Linguagens de Programa\c{c}\~{a}o
%


\chapter{ Conceitos básicos da Linguagem Julia}

%Neste capítulo, buscamos apresentar de forma condensada e reestruturada o conteúdo de preparação de ambiente e tipos de dados presentes nos quatro principais livros da bibliogradia desse trabalho. 

Neste capítulo apresentamos os conceitos basais de sintaxe, design e ferramental necessários para se programar em Julia. Nesse sentido, todo o conteúdo dete capítulo foi baseado nos quatro principais livros-texto da bibliografia desse trabalho. Sendo eles, principalmente \cite{Lobianco2019} e \cite{Balbaert2016}, seguidos de \cite{Lauwens2019} e \cite{Kwong2020}.

\section{Preparação do ambiente}
%Faltando algo para conectar esse parágrafo ao anterior []
Para executar código Júlia, basta baixar e descompactar os arquivos binários no site oficial da linguagem. O arquivo executável apresenta o console de interpretação mostrado na figura \ref{REPL} (também conhecido como "REPL - Read, Eval, Print, Loop") por onde é possível executar Julia tanto por linha de comando quanto por meio de scripts .jl. Também é possível utilizar diversos ambientes de desenvolvimento integrado que serão abordados alguns capítulos adiante.
%Deveria colocar esse parágrafo em uma nova section? tipo first things first? []
\begin{figure}[H]
\begin{center}
    \caption{REPL Julia} \label{REPL}
    \includegraphics[width=12cm]{aplicacoes/ambiente.png} \\
    {\tiny \sf Fonte: Autor}
\end{center}
\end{figure} 

\subsection{Pacotes e módulos}

Os criadores da lingua escolheram construir um núcleo leve complementado por uma biblioteca padrão (Standard Library) distribuída junto dos binários, e um poderoso gerenciador de pacotes capaz de baixar (muitas vezes direto de repositórios GitHub), pré compilar, atualizar e resolver dependências de pacotes a partir de comandos simples. 

Para acessar o gerenciador de pacotes podemos importar o módulo de pacotes (import Pkg) e executar os comandos no formato Pkg.(ARGS). Demonstramos esse caso de uso na figura \ref{import_Pkg} onde executamos o comando para instalar o pacote Plots. Como o pacote já está instalado, o Pkg apenas o reconhece e não altera os pacotes.
%A figura \ref{import_Pkg} demonstra esse caso de uso ao tentarmos instalar o pacote Plots que como já está instalado, podemos observar que o Pkg reconhece tal fato e não altera nossos pacotes. 
Também podemos utilizar o modo especial de pacotes no REPL que traz algumas facilidades como autocompletar por exemplo. Nesse caso, basta digitar "]" para entrar nesse modo especial do REPL, como mostrado na figura \ref{pkg}. Na figura \ref{REPL_plots} repetimos o comando para instalar a biblioteca Plots, dessa vez no REPL. 
%Já no REPL também podemos apenas digitar ] para entrar no modo especial de pacotes, conforme a figura \ref{pkg}. nele temos algumas facilidades como autocomplete.
\begin{figure}[H]
\begin{center}
    \caption{Utilizando o Pkg em scripts} \label{import_Pkg}
    \includegraphics[width=12cm]{aplicacoes/import_Pkg.png} \\
    {\tiny \sf Fonte: Autor}
\end{center}
\end{figure} 

\begin{figure}[H]
\begin{center}
    \caption{REPL - modo pacotes} \label{pkg}
    \includegraphics[width=12cm]{aplicacoes/pkg.png} \\
    {\tiny \sf Fonte: Autor}
\end{center}
\end{figure} 

\begin{figure}[H]
\begin{center}
    \caption{Utilizando modo Pkg do REPL } \label{REPL_plots}
    \includegraphics[width=12cm]{aplicacoes/REPL_plots.png} \\
    {\tiny \sf Fonte: Autor}
\end{center}
\end{figure}

Listamos a seguir alguns dos principais comandos do módulo Pkg. Em seguida demonstramos exemplos com o comando "status"  tanto no modo especial de pacotes do REPL (figura \ref{pacotes}) quanto em um exemplo de código Julia (figura \ref{Pkg_script}):
\begin{itemize}
  \item \textbf{status:} retorna uma lista (nome e versão) dos pacotes instalados localmente.
  \item \textbf{update:} atualiza o índice local de pacotes e os próprios pacotes para a versão mais recente.
  \item \textbf{add nomePacote:} automaticamente baixa e instala um pacote.
  \item \textbf{rm nomePacote:} remove o pacote e todas as suas dependências.
\end{itemize}

\begin{figure}[H]
\begin{center}
    \caption{Pkg.status no modo de pacotes do REPL} 
    \label{pacotes}
    \includegraphics[width=12cm]{aplicacoes/pacotes.png} \\
    {\tiny \sf Fonte: Autor}
\end{center}
\end{figure} 
\begin{figure}[H]
\begin{center}
    \caption{Pkg.status em código Julia} 
    \label{Pkg_script}
    \includegraphics[width=12cm]{aplicacoes/Pkg_script.png} \\
    {\tiny \sf Fonte: Autor}
\end{center}
\end{figure} 

Uma vez instalado, para utilizar um pacote devemos utilizar um dos comandos seguintes:
\begin{itemize}
  \item \textbf{use:} permite acessar diretamente as funções de um pacote, basta adicionar o comando (using nomePacote) no inicio do script. 
  \item \textbf{import:} tem a mesma funcionalidade do use com a diferença de precisarmos nos referir ao nome completo da função (nomePacote.nomeFunção), o que tem a vantagem de manter a organização dos nomes. Via de regra, nesse caso definimos apelidos para os pacotes como por exemplo chamar o "Plots" de "pl" (const pl = Plots) %adicionar exemplo de código? []
\end{itemize}

%%Da mesma forma podemos usar (\textbf{use}) ou importar (\textbf{import}) qualquer arquivo Julia como módulo. Ele será analogamente executado ao ser invocado e qualquer símbolo definido estará disponível no escopo onde foi chamado. 

\subsection{Sistema de ajuda}
Além do sistema de pacotes, basta digitar "?"  para acessar o modo especial de ajuda do REPL. Assim, acordo com a documentação do termo buscado o sistema de ajuda pode retornar sua lista de métodos, exemplos de uso, descrição, lista de argumentos ou termos ou ainda os termos relacionados, conforme podemos ver na figura \ref{ajuda}. 
    \begin{figure}[H]
    \begin{center}
        \caption{REPL - modo ajuda} \label{ajuda}
        \includegraphics[width=12cm]{aplicacoes/help.png} \\
        {\tiny \sf Fonte: Autor}
    \end{center}
    \end{figure} 
%a linguagem também oferece um sistema de ajuda que retorna informação sobre o uso da maior partes das funções que pode ser acessado no REPL digitando "?" ou através do comando "?{termo de busca}". 

\subsection{Considerações gerais sobre sintaxe}
Ainda na preparação do ambiente para se programar, a sintaxe da linguagem é um ponto fundamental. Nesse sentido, listamos alguns desses aspectos gerais que consideramos importantes para se começar a programar em Julia: 

%é importante conhecer certos aspectos gerais da sintaxe da linguagem escolhida. A seguir, listamos as caracteristicas sintáticas gerais que julgamos serem mais relevantes para se começar a programar em Julia. 

%em Julia, alguns aspectos de sua sintaxe são importantes de se considerar. A seguir listamos os que achamos mais relevantes ao escopo do trabalho e futuos capítulos 

%Antes de abordamos as estruturas de dados e suas 
%A seguir listamos alguns pontos importantes da sintaxe de Julia. 
%Adicionar exemplos em código? []
\begin{itemize}
    \item Podemos adicionar ao código comentários de uma linha (\#) ou de múltiplas linhas (\#= =\#) que podem ser aninhados e colocados em qualquer lugar como mostra a figura \ref{comentarios}.

    \item Blocos não precisam de parênteses, utilizamos a palavra "end" para indicar o fim de um bloco, conforme figura \ref{blocos_identacao}.
    \item Espaços são sintaticamente significantes, identação, não. (Fig \ref{blocos_identacao}).
    \item Nomes de variáveis aceitam um subconjunto dos simbolos Unicode como letras gregas e até emojis. (Fig \ref{emoji})
    \item Vetores iniciam com o índice 1. (Fig \ref{indice})
    \item Por convenção, funções que podem alterar algum de seus argumentos têm um ponto de exclamação (!) no fim de seu nome. A função de ordenação (sort), por exemplo, recebe um vetor V desordenado e tem duas possibilidades: a função sort(V) retorna um vetor com os elementos de V ordenados, sem alterar o vetor V. Já a função sort!(V) ordena o próprio vetor, sem retorno. Podemos observar essa diferença na figura \ref{exclamacao}.
    \item O ponto e vírgula (;) é utilizado para suprimir o output de um comando ou acessar o shell (a partir do REPL) como apresentado na figura \ref{supressao}.
    
  \end{itemize}
    
    \begin{figure}[H]%comentarios
    \begin{center}
        \label{comentarios}
        \caption{Comentários em Julia} 
        \includegraphics[width=12cm]{aplicacoes/comentarios.png} \\
        {\tiny \sf Fonte: Autor}
    \end{center}
    \end{figure} 

    \begin{figure}[H]%blocos
    \begin{center}
        \label{blocos_identacao}
        \caption{Blocos e identação em Julia} 
        \includegraphics[width=12cm]{aplicacoes/bloco_identacao.png} \\
        {\tiny \sf Fonte: Autor}
    \end{center}
    \end{figure} 

%    \begin{figure}[H]%identacao
%    \begin{center}
%        \label{identacao}
%        \caption{Identação em Julia} 
%        \includegraphics[width=12cm]{aplicacoes/identacao.png} \\
%        {\tiny \sf Fonte: Autor}
%    \end{center}
%    \end{figure} 
 
    \begin{figure}[H]%supressao
    \begin{center}
        \label{supressao}
        \caption{Supressão de output em REPL} %\label{REPL}
        \includegraphics[width=6cm]{aplicacoes/supressao.png} \\
        {\tiny \sf Fonte: Autor}
    \end{center}
    \end{figure} 

    \begin{figure}[H]%emoji
      \begin{center}
        \label{emoji}
        \caption{Unicode nos nomes de variável} 
        \includegraphics[width=6cm]{aplicacoes/emoji.png} \\
        {\tiny \sf Fonte: Autor}
    \end{center}
    \end{figure} 

    \begin{figure}[H]%indice
    \begin{center}
        \label{indice}
        \caption{Indices em Julia iniciam na posição 1} 
        \includegraphics[width=12cm]{aplicacoes/indice.png} \\
        {\tiny \sf Fonte: Autor}
    \end{center}
    \end{figure} 

    \begin{figure}[H]%exclamacao
    \begin{center}
      \label{exclamacao}
      \caption{Diferença entre as funções sort() e sort!()} 
        \includegraphics[width=12cm]{aplicacoes/exclamacao.png} \\
        {\tiny \sf Fonte: Autor}
    \end{center}
    \end{figure} 
\section{Tipos e estruturas de dados}
Segundo \cite{Lobianco2019}, Julia oferece nativamente um sistema hierárquico e bastante completo de tipos predefinidos que são divididos entre \textbf{escalares} e \textbf{coleções}.
Escalares são tipos atômicos, indivisíveis, como por exemplo inteiros, floats, e chars. Já as coleções são objetos que acomodam outros objetos como por exemplo vetores multidimensionais, dicionários e conjuntos. 
Nesta seção trazemos as ideias de \cite{Lobianco2019} e \cite{Balbaert2016} para organização e conteúdo dos diferentes tipos e estruturas de dados em Julia.

Nesse sentido, podemos destacar as seguintes características: 
\begin{itemize}
    \item Cada valor (mesmo os primitivos) tem seu tipo único, por convenção iniciando com letra maiúscula como Int64 e Bool (fig \ref{tipos_escalares}). 
    \item Na hierarquia dos tipos, o mais amplo é o tipo genérico Any.
    \item No caso de coleções e alguns escalares, o nome de seu tipo é seguido de chaves indicando os tipos dos elementos contidos, e sua dimensão, como por exemplo Array\{Int64,1\}, normalmente chamado de Vector\{Int64\}, ou seja uma coleção unidimensional de elementos do tipo Int64 como mostra a figura \ref{tipos_parametricos}. Na terminologia da linguagem são chamados tipos paramétricos.  
    \item Em Julia não existe divisão entre objetos e não-objetos pois todos os valores são objetos tipados, enquanto variáveis são apenas nomes relacionados a valores, portanto não têm tipo. 
   % \item Objetos do tipo coleção podem são denominados imutáveis quando não é possível alterar o seu valor (e valor de seus elementos), 
    \item Podemos converter objetos atraves da função "convert(T,x)", ou passar o objeto como argumento para uma função do tipo no formato "T(x)" como "Int64(x)" por exemplo, conforme a figura \ref{conversao} demonstra. 
    \item O operador :: pode ser utilizado para anexar anotações de tipo para expressões e variáveis conforme a figura \ref{anotacao_tipo}. \cite{Bezanson2017} afirma que essa anotação manual de tipos é totalmente opcional, pois apesar de haver situações nas quais ela se traduz em algum ganho de performance, na imensa maioria dos casos a inferência de tipos é suficiente para atingir eficiência máxima na compilação. Ainda assim a anotação de tipos é frequentemente utilizada para confirmar que um programa se comporta conforme o esperado. \label{anotacao_tipos}
    
\end{itemize}


    \begin{figure}[H]
    \begin{center}
        \caption{Tipos escalares de dados} \label{tipos_escalares}
        \includegraphics[width=10cm]{aplicacoes/tipos_simples.png} \\
        {\tiny \sf Fonte: Autor}
    \end{center}
    \end{figure} 

    \begin{figure}[H]
    \begin{center}
        \caption{Tipos paramétricos de dados} \label{tipos_parametricos}
        \includegraphics[width=12cm]{aplicacoes/tipos_parametricos.png} \\
        {\tiny \sf Fonte: Autor}
    \end{center}
    \end{figure}

    \begin{figure}[H]
    \begin{center}
        \caption{Conversão de tipos} \label{conversao}
        \includegraphics[width=12cm]{aplicacoes/convert.png} \\
        {\tiny \sf Fonte: Autor}
    \end{center}
    \end{figure}


    \begin{figure}[H]
    \begin{center}
        \caption{Anotação de tipos} \label{anotacao_tipo}
        \includegraphics[width=12cm]{aplicacoes/type_annotation.png} \\
        {\tiny \sf Fonte: Autor}
    \end{center}
    \end{figure} 

    \subsection{Tipos simples}
    \begin{itemize}
      \item Caracteres individuais são do tipo Char, e representado por aspas simples.
      \item Booleanos são do tipo Bool, apenas com as instâncias True e False. Em um contexto de inteiros, booleanos podem ser interpretados como 0 e 1, assim como os inteiros 0 e 1 podem ser interpretados como booleanos, conforme é demonstrado na figura \ref{bool}.  
      %porém o oposto não ocorre (if 0 gera o erro de tipo "non-Boolean used in Boolean context"
    \item O tipo padrão de inteiro é o Int64 (existem outros 9 tipos inteiros) capaz de armazenar valores entre $-2^{63}$ e $2^{63-1}$.
    \item De modo análogo o padrão de ponto flutuante é o Float64.
    \item Números complexos são suportados pela variável global im que representa a raiz quadrada de -1 (fig \ref{complex}).
  \end{itemize}
 
    \begin{figure}[H]
    \begin{center}
        \caption{Tipo Bool} \label{bool}
        \includegraphics[width=8cm]{aplicacoes/bool.png} \\
        {\tiny \sf Fonte: Autor}
    \end{center}
    \end{figure} 

  \begin{figure}[H]
  \begin{center}
      \caption{Números complexos} \label{complex}
      \includegraphics[width=12cm]{aplicacoes/complex.png} \\
      {\tiny \sf Fonte: Autor}
  \end{center}
  \end{figure} 

\subsection{Operações básicas}
Além dos operadores padrões de soma, subtração, multiplicação e divisão (+, -, *, /) temos a potenciação através do circunflexo ("3\^{}2"), a raiz quadrada por meio da função "sqrt()", o resto pelo operador \%, e a divisão inteira pelo símbolo $\div$ ("\textbackslash div" + tab), conforme apresentamos na figura \ref{operacoes}. 

    \begin{figure}[H]
    \begin{center}
        \caption{operacoes} \label{operacoes}
        \includegraphics[width=12cm]{aplicacoes/operacoes_basicas.png} \\
        {\tiny \sf Fonte: Autor}
    \end{center}
    \end{figure} 



\subsection{Arrays}
Arrays (Array\{T, n\}) são coleções indexadas de objetos de tipo T, com n dimensões. %Isto é, um array acomoda elementos em ordem 
 Eles podem conter elementos de apenas um tipo (T), sendo que T pode ser de algum dos tipos que já abordamos (como inteiros e floats) ou dos tipos especiais \textbf{Any} ou \textbf{Union{}}.
 
 O tipo genérico Any tem a vantagem de permitir arrays genéricos Array\{Any,n\} com elementos de quaisquer tipos diferentes, entretanto, de acordo com \cite{Lobianco2019} são consideravelmente menos performáticos que arrays de tipos definidos. Isto porque para tipos definidos o compilador da linguagem é capaz de gerar código específico e otimizado para os tipos em questão, até mesmo para arrays do tipo Union\{$T_1,...,T_n$\}, que comportam elementos de cada um dos tipos $T_i$ definidos. O que não é possível para arrays do tipo Any.  
 %heterogêneos -nesse caso tendo o tipo genérico (Any) que em geral é consideravelmente menos performático- ou ser do tipo união (Union\{$T_1,T_2$\}) de outros tipos (exemplo: Union\{Int64,String\}), que consegue ter performace próxima do monotipo a partir dessa restrição.
 Alguns tipos de arrays recebem nomes especiais como por exemplo os vetores (Vector\{T\}) que são vetores unidimensionais, as matrizes (Matrix\{T\}) que são vetores bidimensionais e as Strings que são ainda um subtipo de vetor de caracteres. 

%Além das Strings que são vetores imutáveis de caractéres, também temos os tipos especiais Vector\{T\} e Matrix\{T\} que são respectivamente Arrays de uma e duas dimensões.

\subsubsection{Criação de Array}
Existem diversas formas de criarmos um Array do tipo T:
\begin{lstlisting}
     a  = [1;2;3] #cria um vetor coluna de uma dimensao.
     a = [1 2 3] #cria um vetor linha de uma dimensao.
     a = [] #cria um Array{Any,1}
     a = Int64[] #constroi um vetor de Int64 com uma dimensao.
     a = Array{Int64,1}() #mesmo do anterior, usando construtor.
     a = zeros(n) #cria um Array{Float64,1}
     a = zeros(Int64,n) #cria um array do tipo Int64 com n zeros.
     a = fill(j,n) #vetor com n elementos j
     a = rand(n) #vetor preenchido com n numeros aleatorios
\end{lstlisting}

\subsubsection{Acessar elementos}
Podemos acessar subvalores de um array utilizando chaves para selecionar um elemento (a[2]) ou uma fatia de intervalo fechado (a[de:passo:até]) da seguinte forma:
\begin{lstlisting}
   a = [1 2 3 4 5 6 7] 
   a[4] #retorna 4
   a[1:3] #retorna os elementos 1,2,3
   a[1:2:end] #retorna 1 3 5 7
   a[end:-1:1] #retorna 7 6 5 4 3 2 1
\end{lstlisting}

\subsubsection{Principais funções}
A seguir apresentamos as principais função para trabalharmos com Arrays, é interessante notar que normalmente as funções que podem modificar algum de seus argumentos contém uma exclamação.
\begin{lstlisting}
  push!(a,b) #adiciona o elemento b no fim de a.
  append!(a,b) #adiciona os elementos de b em a,
  #identico ao push! caso b seja escalar

  c = vcat(1,[2,3],[4,5]) #concatena arrays

  pop!(a) #remove o ultimo elemento do array,
  popfirst!(a) #remove o primeiro elemento do array.
  deletat!(a,pos) #remove o elemento da posicao pos

  pushfirst!(a,b) #b no inicio de a.

  length(a) #=ou=# if a in b end #retorna o comprimento do vetor

  sort!(a) #ordena o vetor a.
  sort(a) #retorna a ordenado sem modificar o vetor original
  reverse(a) #retorna elementos de a invertidos

  unique!(a) #remove duplicatas modificando a.
  unique(a) #retorna a sem duplicadas. 
  in(b,a) #checa a existencia de b em a.

  a... #=operador "splat" coverte os elementos em 
  parametros para uma funcao.=#

  maximum(a) #=ou=# max(a...) #retorna o maior valor.
  minimum(a) #=ou=# min(a...) #analogo ao anterior.
  sum(a) #retorna a soma dos elmentos de a.
  cumsum(a) #retorna um vetor com a soma cumulativa de a.

  empty!(a) #esvazia um vetor coluna.
  b = vec(a) #transforma vtores linha em vetores coluna.
  shuffle(a) #=ou=# shufle!(a) #=embaralha aleatoriamente 
  os elementos de a.=# 
  (requer o modulo Random).=#

  isempty(a) #checa se um array esta vazio.

  findall(x ->  x == value, a) #=retorna o indice de todas
  as ocorrencias de x.=#
  deleteat!(a, findall(x ->  x == value, a)) #=deleta todas as 
  ocorrencias de x de a.=#

  enumerate(a) #retorna um iterador de pares (indice, elemento).
  zip(a,b) #retorna um iterador de pares (a_element,b_element).
\end{lstlisting}

\subsubsection{Arrays aninhados e multidimensionias}
Arrays multidimensionais são os objetos do tipo Array\{T,N\} sendo o número de dimensões N maior que um, enquanto Arrays Aninhados apresentam uma dimensão sendo pelo menos um de seus elementos outro Array, como por exemplo um vetor de vetores Array\{Array\{T,1\},1\}.

A principal diferença entre ambos é que em matrizes o número de elemento em cada coluna deve ser igual e as regras da Álgebra Linear são aplicáveis. 


Os elementos de Arrays aninhados podem ser acessados por chaves duplas (a[2][3]), Já os elementos de multidimensionais podem ser acessados com os índices de cada dimensão separados por vírgula (a[lin,col]). Para vetores colunas tanto a[2] quanto a[1,2] retornam o segundo elemento. 

Podemos contruir Arrays multidimensionais de modo análogo aos vetores, alias estes são um caso específico daqueles:
\begin{lstlisting}
  a = [[1,2,3] [4,5,6]] #cria por colunas
  a = [1 4; 2 5; 3 6] #cria por linhas
  a = zeros(In64,n,m,g) #=cria uma matriz nxmxg 
  preenchida com zeros.=#
  a = [3x + 2y + z for x in 1:2, y in 2:3, z in 1:2] 
  #=tambem podemos usar list comprehension.=#
\end{lstlisting}

Também temos funções particularmente úteis para trabalharmos com Arrays multidimencionais:
\begin{lstlisting}
  size(a) #retorna uma tupla com os damanhos de cada dimencao
  ndimns(a) #retorna o numero de dimensoes.
  reshape(a,nElementDim1, nElementsDim2,...,nElementDimN) 
  #redimensiona o array.
  dropdimns(a, dimns=(dimRemov1, dimRemov2) 
  #remove as dimensoes especificadas
  transpose(a) #=ou=# a' #transpoe vetores ou matrizes.
  hvat(col1,col2) #concatena horizontalmente
  vcat(row1, row2) # concatena verticalmente. 
\end{lstlisting}
\subsection{Strings}
As strings (String) em Julia são vetores imutáveis de caracteres, isto é: conforme apresentamos na figura \ref{string_imutavel}, não é possível alterar nenhum dos elementos de uma String. 
Elas são definidas com aspas duplas e, assim como vetores, suportam indexação e looping.

\begin{figure}[H]
\begin{center}
    \caption{String imutável} \label{string_imutavel}
    \includegraphics[width=12cm]{aplicacoes/string_imutavel.png} \\
    {\tiny \sf Fonte: Autor}
\end{center}
\end{figure} 

Algumas das operações típicas com strings são:
\begin{itemize}
    \item \textbf{split}(s, "\ ") nesse exemplo utiliza o espaço como um separador e retorna um vetor com as sublistas.
    \item \textbf{join}([s1,s2], "\ ") que une strings utilizando, nesse caso, o espaço entre cada junção.
    \item \textbf{replace}(s, {termo buscado} =  {substituto}) que substitui substrings compatíveis com a busca.
    \item \textbf{parse}(Int, "42") retorna a string convertida para um tipo numérico, no caso inteiro.
    \item \textbf{string}(123) retorna a string correspondente ao número.
    \item Existem três principais maneiras de concatenar strings:
        \subitem 
        \begin{lstlisting}    
  string("Hello, ","world")
  "Hello, World! The answer is $(43)"
  "Hello, " * "world"
        \end{lstlisting}
        
\end{itemize}


\subsection{Tuplas e Tuplas Nomeadas}
Tuplas (Tuple\{T1, T2,...\}) são grupos de tamanho fixo que contém valores separados por vírgula. Diferentemente de arrays, os valores de uma tupla não precisam ser do mesmo tipo, e esse fato não diminui sua performace visto que o tipo de uma tupla é uma tupla dos tipos de seus valores, análoga ao tipo Union{} como podemos observar na figura TAL. 
 Outra diferença em relação aos arrays é que tuplas são imutáveis, isto é, seus valores não podem ser modificados após a criação.   
 Podemos criar tuplas com parênteses ou sem conforme o exemplo:
\begin{lstlisting}
  t = (1, 2.5, "a")
  t = 1, 2.5, "a"
#Podemos converter uma tupla em um array:
  a = [t...] #utilizando o operador splat
  a = [i[1] for i in t] #utilizando list comprehension
  a = collect(t) #utilizando o operador collect. 
#De modo analogo podemos converter um array em uma tupla:
  t = (a...,)
\end{lstlisting}  
%\subsection{Tuplas nomeadas}
Já as Tuplas Nomeadas (NamedTuple) são coleções de itens que além do índice seus elementos, cada um deles pode ser identificado com uma chave única da seguinte forma: 
\begin{lstlisting}
  nt = (a=1,b=2.5) #define uma tupla nomeada
  nt.a #acessa o elemento a da tupla nt.
  keys(nt) #returna uma tupla com os nomes: (a,b)
  values(nt) #retorna uma tupla com os valores: (1, 2.5).
  collect(nt) #retorna um array com os valores.
  pairs(nt) #retorna um iterador dos pares (nome,valor) 
\end{lstlisting}

\subsection{Dicionários}
Dicionários (Dict\{Tkey, Tvalue\}) são coleções de valores relacionados a chaves únicas no formato (chave, valor). Apesar do mecanismo parecido com Tuplas Nomeadas, Dicionários são mutáveis e não retornam nem armazenam os pares chave-valor em ordem. 
%guardam mapas de chaves para valores com ordem aparentemente aleatória e diferente das Tuplas nomeadas, são mutáveis, tipo-instáveis.
Os principais métodos para trabalharmos com dicionários são:
\begin{lstlisting}
  d = Dict() 
  #cria um dicionario vazio.
  d = Dict{String,Int64}() 
  #cia um dicionario com tipagem definida para chaves e valores.
  d = Dict('a'= 1,'b'= 2, 'c'= 3) 
  #inicializa o dicionario com valores
  d[novaChave] = novoValor 
  #adiciona um novo par key-value ao dicionario.
  delete!(d,"chave") 
  #deleta o par correspondente a chave.
  map( (i,j) -  d[i]=j, ['a','b','c'],[1, 2 , 3]) 
  #adiciona pares de mapeamentos
  d["chave"] 
  #=retorna o valor correspondente a chave, 
  ou um erro caso ela nao exista.=#
  get(d, 'chave', "Chave inexistente") 
  #=retorna o valor da chave ou um valor padrao 
  caso ela nao exista.=# 
  keys(d) #retorna um iterator com todas as chaves de d.
  values(d) #retorna um iterador com todos os valores de d.
  haskey(d,"chave") 
  #retorna um booleano de acordo com a existencia da chave.
  in(('a'= 1), d) 
  #=checa se o par existe e a chave corresponde 
  ao valor especificado.=#

\end{lstlisting}

\subsection{Sets}
Usamos sets (Set\{t\}) para representar conjuntos mutáveis e sem ordem de valores únicos.
    \begin{lstlisting}
          s = Set() #=ou=# Set{T}() 
        #cria um set vazio
          s = Set([1,2,2,3,4]) 
        #inicializa com valores desconsiderando o 2 duplicado
          push!(s,5) 
        #adiciona elementos
          delete!(s,1)
        #deleta elementos
          intersect(s1,s2) 
        #intersessao de s1 e s2
          union(s1,s2) 
        #uniao de s1 e s2
          setdiff(s1,s2) 
        #diferenca entre s1 e s2
    \end{lstlisting}

\subsection{Números aleatórios}
Julia também facilita a obtenção de números pseudo aleatórios:
    \begin{lstlisting}
  rand()  
#retorna um float entre 0 e 1
  rand(a:b) 
#retorna um inteiro no intervalo [a,b]
  rand(a:0.01:b) 
#float aleatorio com precisao no segundo digito

#=Tambem podemos obter numeros de alguma distribuicao particular, 
desde que usemos o pacote Distributions=#

  rand(nomeDistribuicao[parametros])
  rand(Uniform(a,b))#por exemplo

#=Igualmente importante e a definicao da semente pseudo aleatoria
que permite termos um script reprodutivel:=#
  Random.seed!(1234) 
#define a semente pseudo-aleatoria para 1234
  Random.seed!() 
#reseta a definicao da semente

  rand(Uniform(a,b),2,3) 
#uma matriz 2x3 de numeros uniformemente aleatorios no 
intervalo [a,b].

    \end{lstlisting}
\subsection{Valores Faltantes}
Julia suporta, segundo \cite{Lobianco2019}, diversos conceitos de falta:
\begin{itemize}
    \item \textbf{nothing} (tipo Nothing) é o valor utilizado para blocos e funções que retornam nada. Conhecido como "null do engenheiro de software"
    \item \textbf{missing} (tipo Missing) representa um valor faltante no sentido estatístico, em que deveria haver um valor, apenas o desconhecemos. De modo que os containers e maioria das operações conseguem lidar eficientemente com esses casos. Também conhecido como "null do cientista de dados"
    \item \textbf{Nan} (tipo Float64) representa o resultado de uma operação que retorna é um "não-número". Similar ao missing no sentido de se propagar silentimente ao invés de gerar um erro. Analogamente Julia também oferece Inf (1/0) e -Inf (-1/0).
\end{itemize}
\section{Controle de fluxo e funções}
Em Julia nós temos as principais estruturas condicionais e repetitivas que encontramos nas linguagens mais populares. %devo citar diretamente os quatro livros novamente??
\subsection{Estrutura em bloco}
A sintaxe do controle de fluxo tipicamente se dá atráves de blocos que se iniciam com uma palavra chave seguida de uma condição (com parêntesis opcionais), e finalizam com a palavra "end" da seguinte forma: 
    \begin{lstlisting}
    <palavra chave>  <condicao> 
        #... conteudo do bloco....
    end 
    \end{lstlisting}
    \begin{lstlisting}
    #Por exemplo:
    for i in 1:5
        print(i)
    end     
    \end{lstlisting}
\subsection{Iteração repetida}
\subsubsection{For e while}
As funções for e while em Julia são muito flexíveis, suportando os contrutos padrões como percorer vetores, \textbf{break} que imediatamente aborta uma sequência e \textbf{continue} que imediatamente pula para a próxima iteração. Além de suportar condições múltiplas como demonstrado no exemplo abaixo. 
\begin{lstlisting}
for i=1:2, j=2:5
	println("i:$i, j:$j")
end
#=temos condicoes multiplas que geram um 
loop aninhado onde j percorre o intervalo 2:5 
para cada elemento i.=#

for i in "string exemplo"
	println(i)
end 
#=i percorrera cada caractere de "string exemplo" =#
\end{lstlisting}
\subsubsection{List comprehension}
List comprehension é essencialmente uma forma compacta de escrever um loop:
\begin{lstlisting}
a = [f(i) for i in [1 2 3]] 
#cria um array com f aplicada a cada elemento de [1 2 3]

[x+2y for x in [10,20,30], y in [1,2,3]]
#cria o array 
[d[i]=value for (i,value) in enumerate(lista)

[estudantes[nome] = idade for (nome,idade) in zip(nomes,idades)]
\end{lstlisting}
\subsubsection{Map}
Já a função Map aplica uma função a uma lista de argumentos.
\begin{lstlisting}
map((nome,idade) ->  estudantes[nome] = idade, nomes, idades)

a = map(f, [1,2,3]) 

a = map(x-> f(x), [1,2,3])
\end{lstlisting}

\subsection{Condicionais}
Condicionais em Julia também segue o design de similaridade com as linguagens que a inspiraram. Conforme demonstramos a seguir:
%Condicionais também são estruturados seguindo o design de simplicidade e similaridade com as linguagens dinâmicas:
\begin{lstlisting}
i = 5
if i == 1 
	println("i = 1")
elseif i == 2
	println("i = 2")
else 
	println("i nao e nem 1 nem 2")
end
\end{lstlisting}

As expressões são avaliadas até que seja possível inferir seu resultado (short circuited).%[] Pesquisar quão comun isso é
\subsubsection{Operador ternário}
Assim como list comprehension é uma forma concisa de escrever um loop, temos no operador ternário uma forma concisa de escrever um condicional:
\begin{lstlisting}
a? b: c
#se a eh verdadeiro execute b, do contrario execute c
\end{lstlisting}

\subsection{Funções}
Funções em Julia são muito flexíveis, podem ser delfinadas tanto em linha como em bloco quanto como funções lambda. Além disso, funções também são objetos, e portanto podem ser atribuídas a novas variáveis, retornadas como valores ou aninhadas:
\begin{lstlisting}
  f(x,y) = 2x+y
#definicao em uma linha
#-------------
  function f(x) 
  	2x + y 
  end
#definicao em bloco
#--------------
  x,y ->  2x + y
#funcao anonima
#--------------
  a = f
  a(5)
#funcao como objeto
\end{lstlisting}

As chamadas de funções em Julia seguem uma convenção conhecida como chamada por compartilhamento, que seria entre chamada por referência (onde um ponteiro a memória da variável é passado) e chamada por valor (onde uma cópia da variável é passada e a função trabalha nessa cópia).

Assim, as funções em Julia trabalham nas novas variáveis locais, conhecidas apenas dentro da função, de modo que atribuir a variável a outro objeto não vai influenciar o valor original, mas se o objeto ligado a variável é mutável a mutação do objeto também será aplicada a variável original:
\begin{lstlisting}
  function f(x,y) 
  	x = 10
  	y[1] = 10
  end
  x = 1
  y = [1,1]
  f(x,y)
#Resultado: x = 1, y = [10, 1]

\end{lstlisting}

Na comunidade Julia se recomenda seguir duas regras em relação a funções:%%[] ler no livro pra melhorar a coesão 
\begin{itemize}
    \item Que contenham todos os elementos necessários para sua lógica (sem acesso a nenhuma variável exceto seus parâmetros e constantes globais.)
    \item Que não alterem nenhuma outra parte do programa, ou seja, que não produza nenhum efeito colaterial além da eventual modificação de algum de seus argumentos. 
\end{itemize}

\subsubsection{Argumentos}
\begin{itemize} %%[] Adicionar exemplos em códigos
    \item Normalmente os argumentos são posicionais, mas temos o operador ponto e vírgula (;) para estabelecer que após o mesmo todos os argumentos sejam especificados por nome (keywords arguments)
    \item Os últimos argumentos podem ser especificados junto com valores padrão.
    %\item Também podemos declarar uma função de múltiplos argumentos.
    \item Funções podem receber um número variável de argumentos. 
    \item Também temos a opção de especificar os tipos dos argumentos.
    Como mencionado na sessão \ref{anotacao_tipos}, a principal razão para limitar os tipos dos parâmetros é para rapidamente evidenciar possíveis bugs. Pois assim, caso uma função seja acidentalmente chamada com um tipo diferente do planejado, ela retorna um erro de tipo ao invés de silentemente continuar a execução, que poderia ocasionar um erro de lógica mais difícil de detectar.
    %\item O retorno de uma função é opcional, podendo retornar um valor que pode ser normalmente retornam o último valor computado, mas podem retornar 
    \item Retornar um valor é opcional e normalmente as funções retornam o último valor computado, que pode inclusive ser de tipos coleção como uma tupla ou um array.
\end{itemize}
\begin{lstlisting}
  f(a,b=1;c=2) = (a+b+c) 
  f(1,c=3) 
#(1+1+3)

  function f(args...)
  	for arg in args
  		println(arg)
  	end
  end

  f(a::In64, b::Int64, c::Int64) = (a+b+c)

  f(a,b) = a*2,b+2
  x,y = f(1,2) 
# x = 2, y = 4
    
\end{lstlisting}

%\subsubsection{Multiple-Dispatch}
%Uma função pode ter múltiplos métodos para diferentes números e %tipos de inputs. Ao ser executada o computador selecionará a %implementação correta de acordo com os parâmetros na chamada da %função. Podemos listar todos os métodos de uma função com a função %methods(f).
%%Detalhar mais esse que é o coração da linguagem